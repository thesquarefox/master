\chapter{CD-Inhalt}
\label{A_DVD}

Inhalt des beiliegenden Datentr�gers: % !!!mit Dateinamen und Verzeichnissen

\begin{itemize}
	\item Ausarbeitung als digitale Version: \\
				Datei "`\textbackslash MaxHoenig\_Masterarbeit.pdf"'
	\item Anleitung f�r das beiliegende Programm: \\
				Datei "`\textbackslash Anleitung\_Dungeongenerator.pdf"'
	\item Anleitung zum Kompilieren des Programms: \\
				Datei "`\textbackslash Anleitung\_Kompilierung.pdf"'
	\item �bersicht �ber die Namenskonventionen, die innerhalb des Programms verwendet wurden: \\
				Datei "`\textbackslash Namenskonventionen.pdf"'
	\item Programmcode in C++ als Projekt f�r Microsoft Visual Studio 2010 mit Service Pack 1: \\
				Ordner "`\textbackslash Programmcode\textbackslash"'
	\item F�r Windows x86 kompiliertes Programm (ben�tigt OpenGL): \\
				Ordner "`\textbackslash Dungeongenerator\textbackslash"'
	\item Irrlicht, Version 1.7.3 (verwendete Version) \cite{W_Irrlicht}:\\
				Ordner "`\textbackslash Extern\textbackslash irrlicht-1.7.3\textbackslash"'
	\item Szeneneditor IrrEdit (erm�glicht die  Erstellung von .irr-Szenen), Version 1.5 \cite{W_IrrEdit}:\footnote{Bugs in Irredit 1.5:
1.) Das Einlesen von Meshes, die Meshbuffer mit mehr als $\sim$ 32k Vertices erzeugen, resultiert in inkorrekten Meshes.
Vermutete Ursache: Verwendung von \emph{signed short} statt \emph{unsigned short} als Datentyp f�r die Vertex-Indices.
%Au�erdem wird nur der erste Meshbuffer dargestellt.
%Im 3D-Modelling-Programm 3ds Max 2010 funktioniert das Einlesen problemlos.
2.) Das �ndern von Materialeigenschaften kann die Texturkoordinaten eines Objekts auf falsche Werte setzen.
}\\
				Ordner "`\textbackslash Extern\textbackslash irrEdit-1.5\textbackslash"'
	\item Modellings der f�r die Subszenen verwendeten Basisgeometrien, Andockgeometrien, Verschlussgeometrien
				und Andockstellen als Autodesk 3DS Max 2010 .max-Dateien: \\
				Ordner "`\textbackslash Modellings\textbackslash"'
\end{itemize}


\subsubsection{Beispieldateien}

Beispieldateien f�r H�hlen, Dungeons und H�hlenfarben befinden als XML-Dateien sich im Verzeichnis 
"`\textbackslash Dungeongenerator\textbackslash"'.

\ \\
Beispiele f�r h�hlenartige L-Systeme und bekannte L-Systeme:
\begin{compactitem}
	\item Abbildung \ref{B_LSys_Hoehlenbasis}: "`LSystem\_Hoehle\_Flach1.xml"'
	\item Abbildung \ref{B_BspDungeons} unten: "`LSystem\_Hoehle\_Flach2.xml"'
	\item Abbildung \ref{B_BspDungeons} oben: "`LSystem\_Hoehle\_Tief1.xml"'
	\item viele der Abbildungen, die eine H�hle von innen zeigen: "`LSystem\_Baum.xml"' \cite[S.25]{B_LindSys}
	(stellt von au�en betrachtet eine Baumstruktur dar, eignet sich aber von innen betrachtet auch als H�hle)	
	\item Drachenkurve: "`LSystem\_Drachenkurve.xml"' \cite[S.11]{B_LindSys}
	\item Waben: "`LSystem\_Waben.xml"' \cite{W_LSys_HexaKolam} %http://jsxgraph.uni-bayreuth.de/wiki/index.php/Hexagonal_kolam
	\item Koch-Kurve: "`LSystem\_Koch-Kurve.xml"'
	\item Kochsche Schneeflocke: "`LSystem\_Kochsche-Schneeflocke.xml"'
\end{compactitem}

%\ \\
%Parameter zur Erstellung von h�hlenartigen, zuf�lligen L-Systemen:
%-> als Voreinstellung
%in Anleitung und dieser Arbeit andere Symbole erw�hnt

\ \\
Beispiele f�r Dungeons:
\begin{compactitem}
	\item Abbildung \ref{B_DungeonBsp1}: Datei "`Dungeon\_Aufsteigend.xml"'
	\item Abbildung \ref{B_DungeonBsp2}: Datei "`Dungeon\_Ring.xml"'
	\item Abbildung \ref{B_DungeonBsp3}: Datei "`Dungeon\_Tief.xml"'
	\item Abbildung \ref{B_DungeonBsp4}: Datei "`Dungeon\_Weit.xml"'
	\item Abbildung \ref{B_DungeonBsp5}: Datei "`Dungeon\_Doppelkugel.xml"'
	\item Dungeon aus Abbildung \ref{B_BspDungeons} oben, etwas modifiziert: "`Dungeon\_Verdreht.xml"'
	\item Abbildung \ref{B_DungeonInnenBsp2}(b): "`Dungeon\_Bizarr.xml"'
\end{compactitem}

\ \\
Farben f�r H�hlen:
\begin{compactitem}
	\item Bernstein : "`Farbe\_Bernstein.xml"'
	\item Eis : "`Farbe\_Eis.xml"'
	\item Rubin : "`Farbe\_Rubin.xml"'
	\item Smaragd : "`Farbe\_Smaragd.xml"'
	\item Amethyst : "`Farbe\_Amethyst.xml"'
\end{compactitem}

%\ \\
%Die Einstellungen der Datei "`Einstellungen\_Standard.xml"' werden bei Programmstart automatisch geladen.


%Gute Generierungsvorgaben f�r L-System-H�hlen
%------
%
%- L-Systeme inkl. Parametern
% - bekannte L-Systeme
% - f�r H�hlen aus K3, 8
% 
%- Dungeons aus Anhang 2
%
%- Materialien
%
%- Generierungsparamerter f�r h�hlenartige L-Systeme
 
