\chapter{Export}
\label{KA_Export}

Der Export des Dungeons erfolgt als (a) .irr-Datei f�r die komplette Dungeon-Szene,
(b) Wavefront OBJ-Dateien f�r die erstellten H�hlen- und Ganggeometrien sowie
(c) Zusatzinformationen zum Aufbau des Sichtbarkeitsgraphen als XML-Datei.
Zur weiteren Verwendung der exportierten Daten m�ssen sich alle im Folgenden genannten Dateien
sowie die in den genutzten Subszenen enthaltenen Texturen und Meshes im gleichen Ordner befinden
und auch so weitergegeben werden.

\ \\
Die Benennung der .irr-Szenendatei erfolgt als: \emph{"`[Dungeonname].irr"'}. \\
Der Aufbau des Szenegraphen der exportierten Dungeon-Szene wird im Folgenden dargestellt.
Die Benennung und ID's zum Wiederauffinden jedes Dungeonbestandteils sind
dabei geschrieben als \emph{(Benennung, ID)}.
Eine ID von -1 bedeutet, dass keine explizite ID vergeben wird, eine ID-Angabe von
0++ bedeutet aufsteigende ID's, beginnend mit 0. \\

\begin{compactitem}
	\item \emph{(Dungeon,-1)} $\Rightarrow$ kompletter Dungeon
	\begin{compactitem}
			\item \emph{(Raeume,-1)} $\Rightarrow$ alle R�ume
				\begin{compactitem}
						\item \emph{([gew�hlte Benennung],0++)} $\Rightarrow$ nacheinander alle R�ume mit ihrer Bezeichnung und ihren ID's
				\end{compactitem}
				
			\item \emph{(GangDetailstufe,0++)} $\Rightarrow$ alle vorhandenen Detailstufen der G�nge
				\begin{compactitem}
						\item \emph{(Gang,0++)} $\Rightarrow$ nacheinander alle G�nge mit ihren ID's
				\end{compactitem}
			\item \emph{(GangAdapter,-1)} $\Rightarrow$  alle vorhandenen Gangadapter
				\begin{compactitem}
						\item \emph{(Adapter0,0++)} $\Rightarrow$ nacheinander alle Adapter an Ende $\vv{P(0)}$ des Gangs mit der zum Gang geh�rigen ID
						\item \emph{(Adapter1,0++)} $\Rightarrow$ nacheinander alle Adapter an Ende $\vv{P(1)}$ des Gangs mit der zum Gang geh�rigen ID
				\end{compactitem}
				
			\item \emph{(Detailobjekte,-1)} $\Rightarrow$  alle vorhandenen Detailobjekte
				\begin{compactitem}
						\item \emph{(DetailobjektAnGang,0++)} $\Rightarrow$ nacheinander alle Detailobjekte des Gangs mit der zum Gang geh�rigen ID
						\begin{compactitem}
								\item \emph{([gew�hlte Benennung],0++)} $\Rightarrow$ nacheinander alle Detailobjekte mit ihrer Benennung, die ID's werden von $\vv{P(0)}$ bis $\vv{P(1)}$ aufsteigend gez�hlt
						\end{compactitem}
				\end{compactitem}
				
			\item \emph{(HoehlenDetailstufe,0++)} $\Rightarrow$ alle vorhandenen Detailstufen der H�hle
				\begin{compactitem}
						\item \emph{(Hoehlenteil,0++)} $\Rightarrow$ nacheinander alle Subnetze der H�hle mit ihren ID's
				\end{compactitem}
	\end{compactitem}	 
\end{compactitem}	 

\ \\
Die Benennung der exportierten H�hlengeometrien erfolgt als: \\
\emph{"`[Dungeonname]\_HoehlenMesh\_[SubnetzID]\_Detailstufe\_[DetailstufenNummer].obj"'}. \\
Die Benennung der exportierten Gangmeshes erfolgt als: \\
\emph{"`[Dungeonname]\_Gang\_[GangID]\_Detailstufe\_[DetailstufenNummer].obj"'}. \\
Die Benennung der exportierten Adaptermeshes erfolgt als:\\
\emph{"`[Dungeonname]\_Gang\_[GangID]\_Adapter\_[0 bzw. 1].obj"'}. \\
Zu den OBJ-Meshes zugeh�rig ist die ebenfalls erzeugte Datei \emph{"`DungeonBasisMaterial.mtl"'}, die ein Standardmaterial enth�lt.

\ \\
Die Benennung der Datei f�r die Zusatzinformationen erfolgt als: \\
\emph{"`[Dungeonname]\_Zusatzinformationen.xml"'}. \\
Der Aufbau der Zusatzinformationen ist Folgender:

\lstset{tabsize = 2, basicstyle=\footnotesize}%,  escapeinside={<!--}{-->)} } %,language=XML}
\begin{lstlisting}
<DungeonInformationen>
	<Raueme>
	  <!--Aufz�hlung aller R�ume:-->
		<Raum ID="[ID des Raums]">
			<GangIDs Gang0Nord="[ID des angeschlossenen Gangs, bzw. -1 f�r keinen]"
				Gang1Ost="[...]" Gang2Sued="[...]" Gang3West="[...]" />
		</Raum>
	</Raueme>
		
	<Gaenge GangBreite="[Breite der G�nge in Voxeln]">
		<!--Aufz�hlung aller G�nge:-->
		<Gang ID="[ID des Gangs]"
			DefinitivBlickdicht="[0 oder 1]">
			<RaumIDs RaumAndockstelle0="[ID des angeschlossenen
				Raums, bzw. -1 f�r H�hle]"	 RaumAndockstelle1="[...]" />
			<Position0 X="[...]" Y="[...]" Z="[...] />
			<Position1 X="[...]" Y="[...]" Z="[...]" />
			<Ableitung0 X="[...]" Y="[...]" Z="[...]" />
			<Ableitung1 X="[...]" Y="[...]" Z="[...]" />
		</Gang>
	</Gaenge>
	
	<Hoehle>
		<!--Aufz�hlung aller H�hlensubnetze:-->
		<HoehlenTeil ID="[ID des Subnetzes]">
			<VoxelMin X="[Min X des zugeh�rigen Voxelgebietes]"
				Y="[...]" Z="[...]" />
			<VoxelMax X="[...]" Y="[...]" Z="[...]" />
			<BlickdichtNegativeRichtung X="[kann nicht nach X-negativ
				gesehen werden? 0 oder 1]" Y="[...]" Z="[...]" />
			<BlickdichtPositiveRichtung X="[...]" Y="[...]" Z="[...]" />
		</HoehlenTeil>
	</Hoehle>	
</DungeonInformationen>	
\end{lstlisting}

%\subsubsection{Formate}
%- Export: Dateien als Archiv exportieren
%* 3D-Modelle (Levelst�cke)
%* eventuell Texturen (wenn keine Standardtexturen)
%* zus�tzliche Daten (Abstandsmatrix etc.)
%* bei Zusatzobjekte (Detail, Interaktion) speichern in welchem Gebiet
%* Szenendatei (Irrlicht)
%- Detailobjekte per ID-Nummer und Name im Szenegraph auffindbar -> 
%Verwendung als Interaktionsobjekte wie in \cite{T_BachelorHoenig}
%- XML-Beschreibung (DTD)

\subsubsection{Weitere Exportm�glichkeiten}

Die H�hle l�sst sich auch einzeln in einer OBJ-Datei mit frei w�hlbarem Dateinamen exportieren.
Alle H�hlensubmeshes werden dabei in diese Datei geschrieben.
Die Datei \emph{"`DungeonBasisMaterial.mtl"'} wird hier ebenso erzeugt und ist zur OBJ-Datei zugeh�rig.

\ \\
Die aus L-Systemen erzeugten Turtle-Grafik-Zeichenanweisungen lassen sich ebenfalls exportieren.
Als Dateiformat wird XML verwendet.
Die zum Zeichnen relevanten Parameter werden ebenfalls gespeichert.
Der Dateiaufbau gestaltet sich wie folgt:

\begin{lstlisting}
<LSystemAbleitungen>
	<LSystemParameter>
		<Winkel WinkelGier="[...]" WinkelNick="[...]" WinkelRoll="[...]" />
		<Radius StartRadius="[...]" RadiusFaktor="[...]"
			RadiusDekrementor="[...]" />
	</LSystemParameter>
	
	<!--alle generierten Ableitungen, beginnend mit Iteration 0:-->
	<Iteration Nummer="[Iterationsstufe]"
		String="[Generierter String]" />		
</LSystemAbleitungen>		
\end{lstlisting}